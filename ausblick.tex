\chapter{Ausblick}



\section{Kryo-Elektronenmikroskopie}

In dieser Arbeit wurde mit einem relativ kleinen Datensatz an \ac{SNP}s gearbeitet, die Gründe dafür wurden in Kapitel \ref{sec:probleme_mit_dem_datensatz} diskutiert. Als Hauptgrund hat sich herausgestellt, dass es einfach an der Tatsache liegt, dass zu wenige 3D Strukturen aufgeklärt sind. 

Die Anzahl aller strukturell aufgeklärten Proteine hat sich in den letzten 10 Jahren von 30\% auf 40\% erhöht, im Gegensatz dazu hat sich die Anzahl an bekannten Sequenzen exponentiell vervielfacht. Basierend auf den aktuellen Trends wird erwartet, dass 55\% der strukturellen Abdeckung innerhalb von 15 Jahren erreicht wird \cite{Khafizov.2014}.

Die neusten Entwicklungen in der Chemie, lassen jedoch hoffen, dass in Zukunft zu erwarten ist, dass sehr viel mehr und vor allem hochauflösende 3D Strukturen zu Verfügung stehen. Denn der diesjährige (2017) Nobelpreis in Chemie\footnote{\url{https://www.nobelprize.org/nobel_prizes/chemistry/laureates/2017/announcement.html}
}, mit dem Titel:
\begin{quote}
    ''Für die Entwicklung der Kryo-Elektronenmikroskopie, für die hochauflösende Strukturbestimmung von Biomolekülen in Lösungen.''
\end{quote}
Von Jacques Dubochet, Joachim Frank und Richard Henderson, beschäftigt sich genau mit diesem Thema. So haben die Herren erfolgreich gezeigt, dass es möglich ist die 3D Struktur hochauflösend mit einem Kryo-Elektronenmikroskop in der Zelle zu bestimmen.

\begin{figure}
    \centering
    \includegraphics[width=.95\textwidth]{images/Verbesserung_der_Aufloesung.jpg}
    \caption{Verbesserung der Auflösung von 2013 bis heute 2017\protect\footnotemark{}.}
    \label{fig:verbesserte_aufloesung}
\end{figure}

Bisher war es notwendig das Protein aufzureinigen um es anschließend in Kristallform zu bringen, da die Röntgenstrukturanalyse nur kristalline Strukturen aufklären kann. 

Die Qualität solcher Strukturen hängt von der Auflösung ab, diese liegt bei 6 bis 2 Angström, das ist links in \ac{Abb} \ref{fig:verbesserte_aufloesung} zu sehen. Die neue Methode ermöglicht es die Strukturen deutlich detailierter darzustellen, wie rechts in der \ac{Abb} gezeigt.

\begin{quote}
    ''Now there's an explosion in research''
    - Peter Brzezinski
\end{quote}

Zusätzlich lassen sich so erstmals Interaktionen zwischen Proteinen direkt unter dem Mikroskop erkennen. Damit kann man auch Proteine in ihren verschiedenen Zuständen beobachten. Das ist auch sehr interessant für diese Arbeit, denn so könnten Energieveränderungen in verschiedenen Konformationen betrachtet werden.
\footnotetext{\url{http://www.blopig.com/blog/wp-content/uploads/2014/04/CootLigand2-624x349.png}}

%" Eine physikalisch korrekte Darstellung molekularer Strukturen erfolgt über die Elektronendichte, welche eine realistische Abbildung des Moleküls in der Realität ist."

%\begin{figure}
%    \centering
%    \includegraphics[width=.95\textwidth]{images/approximation.png}
%    \caption{Dargestellt ist eine Approximation der Atome eines Proteins.}
%   \label{fig:approximation}
%end{figure}



\section{Anwendungsgebiete}

Sollten genug 3D Strukturen zu Verfügung stehen, so sollte es möglich sein alle \ac{SNP}s im Illumina Panel zu annotieren. Generell ist die Annotation mittels \ac{EP}s nicht auf das Illumina Panel begrenzt, so sollte es möglich sein alle \ac{SNP}s zu annotieren

Wie in Kapitel \ref{sec:leuk} gezeigt, ist die 5 Jahres Überlebensrate nicht besonders hoch, da die Rückfall Quote nach der Remission sehr hoch ist. 

Um diesen Rückfall rechtzeitig zu erkennen, ist es denkbar nach der Remission spezielle Patienten spezifische pathogene \ac{SNP}s, in regelmäßigen Abständen, im Blut zu suchen. So könnte man früher intervenieren und somit den Rückfall aufhalten, bevor der Patient wieder unter den Symptomen der \ac{AML} leidet.
%\ac{MDS}
Diese \ac{SNP}s im Blut, lassen sich schon in sehr geringer Konzentration nachweisen, noch bevor der Patient wieder die typischen Symptome zeigt. So spricht man auch von personalisierter Medizin, da das \ac{SNP} Profil für jeden Patienten anders aussehen würde.




\section{Zusammenfassung}

% Konstruktion des Datensatzes

% SNP Auswahl

In dieser Arbeit wurde gezeigt, dass sich \acf{APs} zum Annotieren von \acf{SNP}s in Proteinen eignen. Dafür wurden Proteine mit aufgeklärter 3D Struktur untersucht, indem die Aminosäuresequenz mit \ac{SNP}s mutiert wurde. Danach wurden die 3D Strukturen, der mutierten Sequenzen zusammen mit der Referenzsequenz, mittels Homologie Modellierung ermittelt. Für diese Strukturen wurden anschließend \acf{EP} errechnet, welche danach verglichen wurden.

Dieser Vergleich hat ergeben, dass verschiedene \ac{SNP}s sich unterschiedlich auf das Energieniveau im \ac{EP} auswirken. Des weiteren wurde gezeigt, dass pathogene \ac{SNP}s deutlich weiter 

Nach verschiedenen Klassifizierungsansätzen, wurde gezeigt, dass es am sinnvollsten ist globuläre und alpha helicale Membranproteine getrennt zu betrachten. 

Globuläre Proteine wurden nach der totalen Differenz des Energiewertes am \ac{SNP} klassifiziert. So wurde ein MCC von 0,8 ermittelt, wenn die untere Grenze auf 2 und die obere Grenze auf 7 festgesetzt wird. 

Membran assoziierte Proteine wurden nach der Differenz des Energiewertes am \ac{SNP} klassifiziert, hierbei wurde die untere Grenze auf -2,5 und die obere Grenze auf -0,3 festgelegt. Sind die Werte nun unter oder Über dieser Grenze, ist anzunehmen, dass es sich um ein pathogenen \ac{SNP} handelt. Hiermit wurde ein MCC von 1 errechnet, allerdings ist anzumerken, dass der Datensatz relativ klein ist.

Generell lässt sich festhalten das momentan noch zu wenige 3D Strukturen von Proteinen aufgeklärt sind, dies ist auch der Grund warum mit diesem Ansatz keine \ac{AML} assoziierten \ac{SNP}s aus dem Illumina TruSeq Myeloid Sequenzing Panel annotiert werden können.

Dies wird sich jedoch in den nächsten Jahren ändern, da z.B. der diesjährige Nobelpreis in Chemie für die Entwicklung der Kryo-Elektronenmikroskopie für die hochauflösende Strukturbestimmung von Biomolekülen in Lösungen ausgeschrieben wurde. Damit ist es möglich schneller und günstiger hochauflösende 3D Strukturen zu erfassen, als es bisher der Fall war.



\section{Weiterführende Arbeit}

einzelnes Protein zur Bestätigung
aufgeklärte muta + bakterium rodhopsin

Online Service hosten


Das Programm eignet sich als Basismodul für eine klinisch Anayse-Pipeline die der eigentlichen variantenanalyse nachgeschaltet ist