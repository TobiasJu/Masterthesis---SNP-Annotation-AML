\chapter{Abstract}

\begin{english} %switch to English language rules
New methods make it possible to detect a large number of single nucleotide polymorphisms (SNPs) in the genome of patients. However, only in a few cases it is possible to assess whether an SNP is potentially harmful. In this thesis, the question of the suitability of amino acid residue pseudopotentials (APs) to annotate SNPs was investigated. SNPs are small genetic variations that can cause diseases such as Acute Myeloid Leukemia (AML). The APs are a method of transforming the physicochemical properties of the protein structure into an energy profile (EP).

It has been shown that EPs may have significant differences to the reference if they contain SNPs. As a result, it was possible to classify the SNPs with high quality.

Since the structural information from the Protein Data Bank (PDB) is the basis of the energy profiles, it is currently not possible to use this model for the annotation of SNPs in clinical settings, as too few of the genes involved are structurally clarified.

This could change, however, because this year's Nobel Prize in chemistry goes to a new cryo-electron microscopy method for the determination of 3D structures without having to crystallize the proteins, as has been the case so far. In the near future, more and higher-resolved structures could be available.
\end{english}
