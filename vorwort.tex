\chapter{Vorwort}

\textbf{Das Vorwort ist bei \emph{Mathias} geklaut, ich überlege mir noch was eigenes.}
\begin{quote}
\textit{Proteine sind Makromoleküle, aufgebaut aus Aminosäuren, welche als treibende Kraft in allen
lebenden Zellen anzutreffen sind. Die Aufklärung ihrer Funktionen und Strukturen sind wichtige
Grundpfeiler moderner molekularbiologischer und medizinischer Forschung. Eine der größten
Herausforderungen moderner Forschung in diesen Bereichen, ist die Identifikation der korrekten
Faltung der Proteinstrukturen. Bisherige Analysemethoden und Modelle basieren auf der Analyse
der Abfolge der Aminosäuresequenz und nutzen somit nicht die gesamten Informationen,
welche von der dreidimensionalen Struktur zur Verfügung gestellt werden.
Eine neue Methode der Analyse von Proteinstrukturen ermöglicht es mehr Informationen und somit
mehr biologischen Kontext nutzbar zu machen. Dieser Ansatz beruht auf der Ermittlung von
Aminosäurerest-Pseudopotentialen, welcher die gesamten physikochemischen und wechselwirkenden
Eigenschaften der Aminosäuren in Form von Energiewerte nutzbar macht. In der vorliegenden
Arbeit wurde mittels diesen Ansatzes ein neues Grundlagenmodell geschaffen, welches
molekulare Substitutionen zwischen Aminosäuren beschreibt und im Beschreibung molekularphylogenetischer
Methoden Anwendung findet.}
\end{quote}






