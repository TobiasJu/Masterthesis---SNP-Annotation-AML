\chapter{Ausblick}

Nobelpreis in Chemie 2017, "Für die Entwicklung der Kryo-Elektronenmikroskopie, für die hochauflösende Strukturbestimmung von Biomolekülen in Lösungen." von Jacques Dubochet, Joachim Frank, Richard Henderson. 
\url{https://www.nobelprize.org/nobel_prizes/chemistry/laureates/2017/announcement.html}


"Eine physikalisch korrekte Darstellung molekularer Strukturen erfolgt über die Elektronendichte, welche eine realistische Abbildung des Moleküls in der Realität ist."

\begin{figure}
\centering
\includegraphics[width=.95\textwidth]{images/Verbesserung_der_Aufloesung.jpg}
\caption{Verbesserung der Auflösung von 2013 bis heute 2017}
\label{fig:auflösung}
\end{figure}



\begin{figure}
\centering
\includegraphics[width=.95\textwidth]{images/approximation.png}
\caption{Dargestellt ist eine Approximation der Atome eines Proteins}
\label{fig:approximation}
\end{figure}

\footnote{\url{http://www.blopig.com/blog/wp-content/uploads/2014/04/CootLigand2-624x349.png}}