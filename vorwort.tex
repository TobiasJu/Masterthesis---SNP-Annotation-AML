\chapter{Vorwort}

Das menschliche Genom enthält etwa 3 Milliarden Basenpaare, die sich in den 23 Chromosomenpaaren im Zellkern all unserer Zellen befinden. Jedes Chromosom enthält Hunderte bis Tausende von Genen, welche die Anweisungen zur Herstellung von Proteinen enthalten. Jedes der geschätzten 19.000 - 21.000 Gene\cite{Ezkurdia.2014} im menschlichen Genom produziert durchschnittlich drei Proteine. Proteine sind aus Aminosäuren aufgebaut und die Grundlage des Lebens und Bausteine unseres Körpers. Somit ist es nicht verwunderlich, dass Proteine alle Stoffwechselwege in unserem Körper steuern. Mutationen von Proteinen können eine vielfältige Auswirkung auf den Organismus haben und Ursache vieler Krankheiten sein. Daher ist die Aufklärung ihrer Struktur und Funktion von grundlegendem Interesse für die Medizinische Forschung. 
Durch \ac{NGS} haben wir mehr sequentielle Informationen als je zu Verfügung, dies ist auch der Grund warum bisherige Algorithmen zum Verständnis der Funktion auf die Aminosäure Sequenz und Sekundär Struktur der Proteine zurück greifen. Da Struktur und Form meistens geschlossen interagieren, ist seit jeher die richtige Faltung essentiell für die richtige Funktion. Deswegen wäre es am Besten direkt mit der 3D Struktur zu arbeiten, dies ist jedoch nicht immer möglich. Da entweder die 3D Struktur noch nicht aufgeklärt ist oder es zu Aufwendig ist mit dieser zu rechnen.
Mittels der Theorie der \ac{APs} ist es uns nun erstmals möglich die 3D Struktur ohne großen Rechenaufwand für strukturelle Berechnungen nutzbar zu machen. In der vorliegenden Arbeit wurde versucht mit diesem Ansatz eine Annotation von \ac{SNPs} zu ermöglichen und im Rahmen der klinischen \ac{AML} Forschung zu verwenden.

