
\chapter{Kurzfassung}

In dieser Arbeit wurde der Frage nachgegangen in wie fern sich \acf{APs} eignen um Single Nucleotide Polymorphisms (SNPs) zu annotieren. Bei den \ac{APs} handelt es sich um eine Methode der Transformation der physikochemischen Eigenschaften der Proteinstruktur in ein Energiemaß. So können 3D Strukturinformationen in eindimensionale Energiewerte überführt werden, sogenannte Energie Profile.
%Zusätzlich wurde mithilfe der \ac{Pfam} Datenbank und deren Familien versucht einen Datensatz zu erstellen um... 
%Auf Grundlage der Konservierungen mittels Pfam aussagen über die Pathogenität zu treffen...

Es wurde gezeigt, dass \ac{EP}s Signifikante Unterschiede zur Referenz aufweisen können, wenn diese SNPs beinhalten. Aufgrund dessen war es möglich die \ac{SNP}s, mit einen \ac{MCC} von 0,71 zu Klassifizieren.

Jedoch ist es auf Basis der aktuellen Strukturen in der \ac{PDB} nicht möglich dieses Modell zur Annotation von \ac{SNP}s im klinischen Umfeld zu benutzen, da einfach noch zu wenige der betreffenden Gene strukturell aufgeklärt sind.

Dies könnte sich jedoch ändern, denn der diesjährige Nobelpreis in Chemie, geht an eine neue Kryo-Elektronenmikroskopie Methode zur Aufklärung von 3D Strukturen ohne die Proteine dafür in Kristallform bringen zu müssen, wie es bisher der Fall war. Somit sollten in naher Zukunft mehr und vor allem bessere Strukturen zu Verfügung stehen.

