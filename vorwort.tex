\chapter{Vorwort}

Das menschliche Genom enthält etwa 3 Milliarden Basenpaare, die sich in den 46 Chromosomen im Zellkern all unserer Zellen befinden. Jedes Chromosom enthält Hunderte bis Tausende von Genen, welche die Anweisungen zur Herstellung von Proteinen enthalten. Jedes der geschätzten 19.000 - 21.000 Gene\cite{Ezkurdia.2014} im menschlichen Genom produziert durchschnittlich drei Proteine. Proteine sind aus Aminosäuren aufgebaut und damit die Bausteine unseres Körpers. Proteine sind die Grundlage des Lebens und steuern alle Stoffwechselwege in unserem Körper. Mutationen von Proteinen können eine vielfältige Auswirkung auf den Organismus haben und Ursache vieler Krankheiten sein. Daher ist die Aufklärung ihrer Struktur und Funktion von grundlegendem Interesse für die medizinische Forschung. 
Durch \ac{NGS} hat die Wissenschaft mehr sequentielle Informationen als je zuvor zu Verfügung. Dies ist auch der Grund warum bisherige Algorithmen zum Verständnis der Funktion auf die Aminosäuresequenz und Sekundärstruktur der Proteine zurückgreifen. Da Struktur und Form meistens geschlossen interagieren, ist die richtige Faltung essentiell für die richtige Funktion. So kann eine Variation in der \ac{DNA} dafür sorgen, dass eine Erkrankung wie \acf{AML} entsteht. Leider fehlen aktuell noch Methoden um potentiell pathogene Mutationen aufzuspüren. Die 3D-Struktur direkt zu nutzen ist nicht immer möglich, da entweder die 3D-Struktur noch nicht aufgeklärt ist oder es zu aufwendig ist, mit dieser zu arbeiten. Mittels der Theorie der \ac{APs} ist es möglich, die 3D-Struktur ohne großen Rechenaufwand für strukturelle Berechnungen nutzbar zu machen. In der vorliegenden Arbeit wurde versucht, mit diesem Ansatz eine Annotation von genetischen Variationen, den sogenannten \ac{SNP}, zu ermöglichen.
