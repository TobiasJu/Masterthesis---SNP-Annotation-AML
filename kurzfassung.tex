% das deutsche Abstract
\chapter{Kurzfassung}

Durch neue Methoden ist es möglich, eine Vielzahl von Single Nucleotide Polymorphisms (SNPs), im Genom von Patienten, zu ermitteln. Dabei kann jedoch nur in wenigen Fällen eine Einschätzung erfolgen, ob ein \ac{SNP} potentiell schädlich ist. In dieser Arbeit wurde der Frage nachgegangen, inwiefern sich \acf{APs} eignen, um \ac{SNP}s zu annotieren. \ac{SNP}s sind kleine genetische Variationen, welche Krankheiten wie \acf{AML} verursachen können. Bei den \ac{APs} handelt es sich um eine Methode der Transformation der physikochemischen Eigenschaften der Proteinstruktur in ein Energiemaß. So können 3D Strukturinformationen in eindimensionale Energiewerte überführt werden, sogenannte Energieprofile (EPs).



Es wurde gezeigt, dass EPs signifikante Unterschiede zur Referenz aufweisen können, wenn diese \ac{SNP}s beinhalten. Aufgrund dessen, war es möglich, die \ac{SNP}s mit hoher Güte zu klassifizieren. % mit einem \ac{MCC} von 0,8

Da die Strukturinformationen aus der \ac{PDB} die Grundlage der Energieprofile sind, ist es aktuell nicht möglich dieses Modell zur Annotation von \ac{SNP}s im klinischen Umfeld zu benutzen, da noch zu wenige der betroffenen Gene strukturell aufgeklärt sind.

Dies könnte sich jedoch ändern, denn der diesjährige Nobelpreis in Chemie geht an eine neue Kryo-Elektronenmikroskopie Methode zur Aufklärung von 3D-Strukturen, ohne die Proteine dafür in Kristallform bringen zu müssen, wie es bisher der Fall war. Somit könnten in naher Zukunft mehr und vor allem höher aufgelöste Strukturen zur Verfügung stehen. 

